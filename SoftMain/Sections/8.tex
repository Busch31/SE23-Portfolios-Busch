\section{Conclusion}
The path to creating a baseline of JHotDraw involved integrating the developed features, bug fixes, and enhancements into the main codebase, this was done by
using Git and working from a develop branch of the main repository. The process needed careful planning and execution to ensure compatibility and maintain code quality,
and we had to redo some parts of the process due to unforeseen issues like the feacture traces, so this needed to be down correctly before proceeding with the rest of the
project. In the beginning of the project we had one understanding of how the feacture trace should be done, but we later come to understand that our approach was wrong
and we had to redo the feacture trace again, this we found out was important, since the \textit{Impact Analysis} would lay the basis for each persons feature refactoring
of their given feature.

\subsection{Merging baseline}
The merging of the baseline was done by using Git, and the process was done by first creating a develop branch from the main branch, and then creating a feature branch
for each person in the group, this was done to ensure that each person could work on their own feature without affecting the main codebase. When a feature was done, it was
merged into the develop branch, where it was reviewed by a group member. This process was repeated for each feature, bug fix, and enhancement made.

\subsection{System Testing}

Testing played an important role to ensure the stability and functionality of the system when doing the refactoring of ones feature.

\textbf{Unit Testing:} We implemented unit tests to check the methods which was refactored to ensuring each worked as intended.

\textbf{Integration Testing:} After merging features, integration tests were conducted to check the seamless interaction between different parts of the application which had been refactored.

\textbf{User Acceptance Testing:} User acceptance tests were carried to check that the system would met the end user expectations of a refactored feature.

\subsection{Reflections}

Reflecting on the final baseline creation, there were both things that went well and things that did not went as planned.

\textbf{What Went Well:}

The team worked together as intended, with clear communication, regular updates and helping eachother when needed, contributing to a smooth process.

Effective Use of CI assisted in maintaining code quality and accelerating the development cycle.

\textbf{What went not so well:}

The feature trace was not done correctly in the beginning, which led to a lot of extra work and time spent redoing it and thereby being a bit behind, which put pressure on
our timeline and planning.

Time constraints did become a problem somtimes and it affected the the scope of the feature refactoring at some points, leading to a lesser scope for my own feature.

\subsection{Scope Adjustments}

During the development process, I encountered situations where I had to adjust the scope of my feature. I had to priorities what methods I would refactor
due to time constraints, some less critical methods were choosen not to be refactored to ensure that the more critical methods were refactored correctly first.
In some cases, features were scaled down to their essential functionalities to meet deadlines while maintaining quality.

\subsection{Lessons Learned}

Working on this project has been a valuable learning experience for me, and I have gained a lot of knowledge and skills that I can apply in future projects when it comes
to refactoring others code, features and programs. I have learned alot about version control using Git, and how to use it to manage a codebase with multiple developers,
and how to use Git to merge features into a delevopment codebase to make a stable baseline, before merging into the main itself.
I have also learned how to use CI to ensure code quality and stability, and how to use it to automate the build and test process, which is a very useful tool.
I have also learned to better manage my time and how to prioritize tasks to ensure that the most critical tasks are done first, and how to adjust the scope of a feature,
when actively working on it, to ensure that the most critical parts are done first.

All in all I am very satisfied with what i have learned doing the project, and I am looking forward to applying the knowledge and skills
I have gained in future projects where I will need to do code refactoring again.
