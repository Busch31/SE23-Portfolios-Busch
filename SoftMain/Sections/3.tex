\section{Impact Analysis}
\subsection{Brief Introduction}
The impact analysis is used to understand the implications of changing a specific feature within the \textit{JHotDraw} project.
The project itself will receive the different feature changes at different times, as the development team, consisting of 5 members,
they are all working on different features of the application at their own pace, and some members might be further ahead than others at any given time.
After inserting the feature entry points into the \textit{JHotDraw} project, I was able to get an output of different relevant figures: \textit{Feature-code Characterization,}
\textit{Feature-code Correlation Grid,} and \textit{Feature-Package Correlation Graph.}

The feature entry points I used in this project are the following:
\begin{itemize}
    \item \textit{Tools-display}
    \item \textit{Drag-drop}
          \begin{itemize}
              \item \textit{Pressed}
              \item \textit{Dragged}
              \item \textit{Released}
          \end{itemize}
\end{itemize}

The \textit{}Tools-display, as stated in the \textit{}Concept Location chapter, is a class that references the \textit{JDisclosureToolbar} class. The \textit{Tools-display} is a handler callback that has been added to a button;
when pressed, it will change the visibility of the chosen toolbar section from visible to hidden or vice versa.

The \textit{Drag-drop}, which consists of \textit{Pressed}, \textit{Dragged}, and \textit{Released}, are handlers that are connected with the \textit{PaletteToolbarUI}. They are activated in the following order:
\begin{itemize}
    \item \textit{Pressed}: when a tool is pressed on with the click of a mouse.
    \item \textit{Dragged}: when a tool has been pressed and the mouse moves while the button is still being held down by the user.
    \item \textit{Released}: when the user releases the pressed mouse button again.
\end{itemize}
