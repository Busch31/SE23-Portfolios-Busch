\section{Concept Location}

\subsection{Methodology}
The location of the classes that I identified was based on the following tools, which I used to locate the different classes and the different methods
that I found was relevant for the feature I had chosen \textit{Tools Palette}.
\begin{itemize}
    \item Different search methods such as:
          \begin{itemize}
              \item Find all references.
              \item Go to definition.
              \item Quick search.
              \item Global search.
              \item Keyword search
          \end{itemize}
    \item Tree scaling both up and down with Extension reference.
    \item Removing code to see what functionality it would affect, thereby better understand what the different pieces of code did what and affects.
\end{itemize}
\subsection{Table Content Overview}
The table below provides an easy overview of the different tools and processes I used to locate the different classes that I found relevant for my chosen feature.
\begin{table}[ht]
    \centering
    \begin{tabular}{|p{0.5cm}|p{3.5cm}|p{3.5cm}|p{8cm}|}
        \hline
        \textbf{\#} & \textbf{Domain classes}     & \textbf{Tools used}                                               & \textbf{Comments}                                                                                                                                                                                                                                                                                                                                                                                                                                                                               \\ \hline
        1           & \textit{AbstractToolbar}    & Quick Search\newline Find all references\newline Code removal     & I started by looking at the different abstract classes for the whole project. Here I found the \textit{AbstractToolbar} class which looked like the right abstract class I was looking for when my features name is \textit{Tool Palette.} I then tested with \textit{code removal} to see what it would impact in the toolbar, but I just not see any changes to the behavior of the program itself when running, so I started looking at what the \textit{AbstractToolbar} was extended from. \\ \hline
        2           & \textit{JDisclosureToolbar} & Code removal\newline Extension reference\newline Go to definition & When I look at what \textit{AbstractToolbar} was extended from, I found the abstract class named \textit{JDisclosureToolbar}, I again tried code removal, this time giving my first result. The abstract class \textit{JDisclosureToolbar} is responsible for the \textit{show/hide} feature of the \textit{tool palette,} which I needed for my user story \textit{Display.}                                                                                                                   \\ \hline
        3           & \textit{ToolsToolbar}       & Code removal\newline Extension reference                          & I then went back down the reference tree to see where in what class it would end. I ended up in the class \textit{ToolsToolbar}. Here again with \textit{code removal} I tried to see what the class was responsible for. I found that it was not the full toolbar as my first thought had been, but it was only a part of the whole \textit{tool palette.}                                                                                                                                     \\ \hline
        4           & \textit{PaletteToolbarUI}   & Code removal\newline Keyword Search                               & After I hit a dead end with the \textit{Extension reference} tool method, I tried to do a \textit{global search} on different keywords such as \textit{Tool, UI, Palette, Bar, ToolBar} and other keywords that could be assimilated with my feature. With this tool method I found the \textit{PaletteToolbarUI} class which contained handlers. I tried to remove some of these handlers to see what it would affect.                                                                         \\ \hline
    \end{tabular}
    \caption{Overview of Domain Classes and Tools Used}
    \label{table:domain-classes}
\end{table}


