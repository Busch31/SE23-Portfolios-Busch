\section{Exercise 05: Drupal}


\subsection{Background}

\subsubsection{Which vulnerabilities do you think can be used? Pick two potential vulnerabilitiesand describe them in terms of why you picked them, i.e., date and exploit effect.}
For Exercise 5 on Drupal, I would focus on the drupal\_coder\_exec and drupal\_drupageddon vulnerabilities.
These vulnerabilities stand out due to their high exploitability rankings which for both of them are excellent.

The drupal\_drupageddon vulnerability presents a significant risk due to its nature as a SQL injection exploit.
Given that Drupal is a Content Management System (CMS), the potential impact of this exploit is substantial.
CMSs typically store user-generated content pages in databases that are rendered on websites.
Thus, a successful SQL injection attack could grant attackers access to extensive website data.
The severity of this vulnerability is reflected in its nickname, "drupageddon," indicating the widespread problems it caused when exploited.

Also the drupal\_coder\_exec vulnerability is particularly concerning because it stems from a third party plugin.
This aspect implies limited control by Drupal over the vulnerability, as opposed to an in-house developed component.
The critical nature of this exploit lies in its capability for arbitrary code execution,
offering a potent gateway for malicious actors to penetrate deeper into Drupal's server infrastructure.

Both vulnerabilities highlight critical security challenges in Drupal's system, especially considering their potential impacts and the nature of the weaknesses they exploit.


\subsubsection{For the rest of the tutorial, we will use the vulnerability dubbed \textit{drupageddon}. What is the underlying vulnerability?}
The underlying vulnerability is an SQL injection vulnerability.

\subsubsection{What is so severe about the issue?}
Drupal operates as a Content Management System (CMS), where website content is stored in databases.
Consequently, this setup permits a malicious actor to potentially access the entirety of these databases without proper authorization.


\subsection{Post-Exploitation}

\subsubsection{What are possible activities/aims for the post-exploitation phase?}
Having gained access to the machine, the first step is to secure the gotton access by establishing or compromising a user account.
This allows for continuous server access, enabling us to collect extensive information about the target system.
Additionally, we aim to elevate our privileges on the machine, further broadening our capability to acquire more detailed information.

\subsubsection{Write out the list in the file that has the “User Accounts”?}

\begin{tabular}{|l|l|l|l|}
    \hline
    root              & daemon        & bin               & sys          \\
    \hline
    sync              & games         & man               & lp           \\
    \hline
    mail              & news          & uucp              & proxy        \\
    \hline
    www-data          & backup        & list              & irc          \\
    \hline
    gnats             & nobody        & libuuid           & syslog       \\
    \hline
    messagebus        & sshd          & statd             & vagrant      \\
    \hline
    dirmngr           & leia\_organa  & luke\_skywalker   & han\_solo    \\
    \hline
    artoo\_detoo      & c\_three\_pio & ben\_kenobi       & darth\_vader \\
    \hline
    anakin\_skywalker & jarjar\_binks & lando\_calrissian & boba\_fett   \\
    \hline
    jabba\_hutt       & greedo        & chewbacca         & kylo\_ren    \\
    \hline
    mysql             & avahi         & colord            &              \\
    \hline
\end{tabular}


\subsubsection{How does having a list of user names help?}
Possessing a list of usernames makes the execution of brute force attacks easier by reducing the number of unknowns to guess.
Additionally, this list can be utilized in phishing campaigns, leveraging the available information to enhance the appearance of legitimacy.
In scenarios where users may have reused their usernames, and possibly passwords, across multiple sites,
checking against known password databases could reveal credentials that might grant access to the targeted website or system.


\subsubsection{What do the excellent post exploitation scripts for linux offer?}
The advanced post exploitation scripts for Linux provide a wealth of system information, including details on system versions, directories, and usernames.
Additionally, they offer capabilities for establishing persistent access through various backdoor mechanisms and other functionalities.


\subsection{Reflection}

\subsubsection{What is the main issue with the web server? How did it help selecting potentialexploits?}
The primary concern with the web server is its exposed directory listing.
This vulnerability not only enables the precise location of Drupal files to be visible but also gives access to Drupal via the drupal\_drupageddon exploit.
Such access highly aids in compromising the host machine.

\subsubsection{When opening the drupal web page, you are greeted by a warning. Do you thinkthis is good practice? Why or why not?}

\subsubsection{Given a more restrictive web server configuration, finding the relevant information wouldnt have been that easy. Please check dirbuster, to be found in the “Web Appli-cation Analysis” menu. How could this tool help you finding information? Try it outon the Ubuntu metasploitable VM. Use/ usr/ share/ dirbuster/ wordlists/ directory-list-2.3-medium.txtas dictionary.}

\subsubsection{How can effective spying with tools likedirbusterbe prevented?}

\subsubsection{This attack didnt get us all the way to root. How would you continue the pentest?What would be your next actions?}

\subsubsection{Do you have any specific things in mind you would try to get root access?}

\subsubsection{What makes getting a remote shell so powerful?}

