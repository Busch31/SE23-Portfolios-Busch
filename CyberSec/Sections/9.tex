\section{Exercise 09: Intrusion Detection}

\subsection{Short discussion:Use case of the presented options}
Reflecting on the information gathering methods of the presented solutions, its clear that their effectiveness in a security strategy is as varied as their approaches.

\textbf{Logcheck} automates the filtration of server logs, vital for highlighting security incidents amidst voluminous log data.

\textbf{Extended firewall} logging serves as a watchful eye on network ingress and egress, yet without refined filtering, critical signals may be lost in the noise.

\textbf{SSHGuard} offers automated defense against SSH-related threats, but its reliance on IP-based blocking can be circumvented via IP spoofing techniques.

\textbf{Suricata} takes a broader stroke by inspecting real-time network traffic against predefined rules,
ideal for complex environments but demanding in terms of configuration and resource allocation.

The advantages of these tools is in their automation and specificity, while the disadvantages comes from potential over reliance on them without considering
evasive tactics employed by smart attackers. In a large can complex security strategy, these tools should be layered,
with each addressing specific vulnerabilities and collectively providing a more complete security solution than one standalone solution.